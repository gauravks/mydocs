\chapter{Abstract}

FPGA based accelerators provide the flexibility to adapt to application
characteristics and are becoming popular among the high performance computing
community. The Noctua HPC cluster at the Paderborn Center for Parallel
Computing (PC\textsuperscript{2}) contains 16 high performance nodes
equipped with Nallatech 520N,Stratix 10 FPGA accelerator boards.
These boards add a huge processing speedup capabilities with the Stratix 10
FPGA and provides 4 100/40/25/10G QSFP28 network ports for serial FPGA-to-FPGA
communication.

This thesis first evaluates the possible point-to-point topologies between
FPGAs using the 4 ports by implementing prototypes for 2 different topologies.
The evaluation shows that the point-to-point communication can achieve a 10
times higher communication bandwidth than the existing MPI+PCIe communication.
The point-to-point communication also has a higher efficiency of 99.4\% compared
to 60\% for the MPI+PCIe. The point-to-point communication is then implemented
for \texttt{MIDG2 MPI FPGA} application which uses \acl{DG} method to solve Maxwell’s equation
in time domain using multiple FPGAs. The existing OpenCL kernels are extended to
use the point-to-point communication with 2 and 4 FPGAs and an optimized OpenCL
kernel design is implemented to eliminate overheads due to FPGA to CPU interactions.
The evaluation of the extended designs show a 20\% to 40\% improvement in the
execution time of the application allowing the FPGA design to achieve linear speedup for
scaling over 2 and 4 FPGAs.