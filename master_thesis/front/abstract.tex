\chapter{Abstract}

FPGA based accelerators provide the flexibility to adapt to application
characteristics and are becoming popular among the high performance computing
community. To utilize multiple of these FPGA accelerators in a parallel application,
FPGA-to-FPGA communication is required over the nodes. This is
currently performed by creating multiple copies of the data to be
shared between the FPGA and CPU. This causes increase in the latency
of the communication between the FPGAs and affects the application performance.
This can be improved by using direct point-to-point communication
between the FPGAs. To evaluate the benefits of the point-to-point
communication, this thesis extends the parallel FPGA implementation
of the MIDG2 application to use 4 100/40/25/10G QSFP28 network ports
in the BittWare 520N accelerator boards to perform serial FPGA-to-FPGA communication.

The thesis first evaluates the possible point-to-point topologies between
FPGAs using the 4 ports by implementing prototypes for 2 different topologies.
The evaluation shows that the point-to-point communication can achieve a 10
times higher communication bandwidth than the existing MPI+PCIe communication.
The point-to-point communication also has a higher efficiency of 99.4\% compared
to 60\% for the MPI+PCIe. The point-to-point communication is then implemented
for \texttt{MIDG2 MPI FPGA} application which uses \acl{DG} method to solve Maxwell’s equation
in time domain using multiple FPGAs. The existing OpenCL kernels are extended to
use the point-to-point communication with 2 and 4 FPGAs and an optimized OpenCL
kernel design is implemented to eliminate overheads due to FPGA to CPU interactions.
The evaluation of the extended designs show a 20\% to 30\% improvement in the
execution time of the application allowing the FPGA design to achieve linear speedup for
scaling over 2 and 4 FPGAs.