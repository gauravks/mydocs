\chapter{Conclusion}
\label{cha:Conclusion}

This thesis presents the benefits of using point-to-point communication between
FPGAs in the MIDG2 application for accelerating computation of electric and magnetic
fields using the \acl{DG} method. The \texttt{MIDG2 MPI FPGA} is extended to communicate
the shared data between FPGAs using point-to-point communication in two different
topologies viz. \texttt{within node} and \texttt{fully connected}. Two different
designs are implemented using the point-to-point communication. The first design
(\texttt{MIDG2 FPGA IO channels}) extends the OpenCL kernels to include the serial IO external channels to use
the point-to-point communication between the FPGAs. Second design (\texttt{MIDG2 FPGA Only})
introduces changes in the OpenCL kernel to remove the host application interaction completely.

The topologies are initially evaluated using separate prototypes used to perform
tests with synthetic data and data sizes. The evaluation of the topologies
shows the superiority of the point-to-point communication over the existing
MPI+PCIe communication in terms of the peak bandwidth performance. A maximum
bandwidth of 19.84 GB/s is achieved for point-to-point communication in
\texttt{within node} topology using all the four available channels between
two FPGAs compared to 1.94 GB/s for MIDG2+PCIe communication. The point-to-point
communication are also found to be more efficient as they are able to utilize the
99.2\% of the available bandwidth compared to 65\% utilization for the MPI+PCIe
communication.

The results of the evaluation with extended \texttt{MIDG2 MPI FPGA} also show
reduction in the execution time of the application by 30\% using \texttt{within node} topology
and 20\% using \texttt{fully connected} topology for different mesh sizes ranging
from 1k to 200k mesh element. Additional synthetic test with \texttt{MIDG2 FPGA IO channels}
where higher data size transfers between the FPGAs show a speedup of 2 times for the execution
time with a maximum speedup of 3 for 200k with \texttt{fully connected} topology. The synthetic tests
show that utilizing the point-to-point communication with applications with higher communication
demands or higher communication to computation ratio can benefit a lot from the communication with
IO channels.

The IO channels design for the MIDG2 FPGA implementation improves the application performance
by using the point-to-point communication between the FPGAs. Further improvements are
 possible by restructuring the partitioning
and OpenCL kernel structures easily.

\section{Future Work}

Following updates to the MIDG2 FPGA designs can be performed to optimize the design
further and utilize the benefits of the IO channels completely:

\begin{itemize}
    \item The bottleneck analysis in section \ref{sec:analysis} identified bottlenecks with the
    data processing efficiency in the \texttt{VOLUME} and \texttt{SURFACE} kernel. The computation structure
    in the compute can be updated to increase the computation efficiency of kernels to achieve
    higher occupancy. This will allow to utilize the complete global memory bandwidth.
    \item The partitioning scheme of the meshes can be updated to create smaller partitions with
    higher communication to computation ratio. This will allow to utilize the complete
    benefit of the point-to-point communication and further reduce the execution time.
\end{itemize}



