\documentclass[]{ccs-thesis}
% options:
% [germanthesis] - Thesis is written in German
% [plainunnumbered] - Don't print numbers on plain pages
% [earlydraft] - Settings for quick draft printouts
% [watermark] - Print current time/date at bottom of each page
% [phdthesis] - switch to PhD thesis style
% [twoside] - double sided
% [cutmargins] - text body fills complete page


% Author name. Separate multiple authors with commas.
\author{Gaurav Kumar Singh}
\birthday{10. November 1989}
\birthplace{Danapur}

% Title of your thesis.
\title{Clustering in Vehicular Networks}

% Choose one of the following lines. Feel free to change the word "Informatik" to match your degree program.
%\thesistype{Masterarbeit im Fach Informatik}\thesiscite{Master's Thesis~(Masterarbeit)}
%\thesistype{Bachelorarbeit im Fach Informatik}\thesiscite{Bachelor Thesis~(Bachelorarbeit)}
\thesistype{Seminararbeit im Fach Informatik}\thesiscite{Seminar Thesis~(Seminararbeit)}

% List of advisors, separated by commas.
\advisors{Christoph Sommer, Falko Dressler}

% List of referees, separated by commas.
\referees{Christoph Sommer, Falko Dressler}


% Define abbreviations used in the thesis here.
\acrodef{WSN}{Wireless Sensor Network}
\acrodef{WANET}{Wireless Ad Hoc Network}
\acrodef{MANET}{Mobile Ad Hoc Network}
\acrodef{VANET}{Vehicular Ad Hoc Network}
\acrodef{CH}{Cluster Head}
\acrodef{CM}{Cluster Member}
\acrodef{CHP}{Cluster Head Pending}

\begin{document}

\pagenumbering{roman}

\maketitle

\chapter*{Abstract}
\addcontentsline{toc}{chapter}{Abstract}
\begin{otherlanguage*}{american}

This thesis captures an overview of ideas, techniques, results and future possibilities of clustering in vehicular networks.
Clustering is a technique to group nodes based on a selected criteria which defines certain level of similarities among the nodes.
Grouping the nodes together in such a way helps define or design a set of functionalities applicable only to the
group and can be applied to the smaller sub-set. In a \ac{VANET} environment clustering presents possibilities
to group vehicles based on a parameter of interest and help to reduce the network traffic, achieve better network throughput,
effective information dissemination.

The thesis presents a set of parameter and respective methodologies based on them for \ac{VANET}s as a comparative study.
First chapter presents the motivation behind the clustering and outlines the basic set of problems which is presented by vehicular
Networks which the researches are trying to address. The second chapter describes the methodologies grouping them based on the main
parameters used for clustering. The third chapter introduces the evaluation techniques used to evaluate the methods along with some
important metrics used to compare the effectiveness of the algorithms. The fourth chapter captures a comparative study based on
the results, highlighting benefits and drawbacks of different approaches based on simulation results. Finally the thesis captures
some ideas which will give an overview of the future research work on the topic.
 
 
\end{otherlanguage*}


\acresetall

\cleardoublepage
\tableofcontents

\cleardoublepage
\pagenumbering{arabic}

\chapter{Introduction}
%\chapter{Einleitung}
\label{sec:introduction}

Along with the advancement in wireless networking in the past two decades, there has been a lot of research targeted
towards developing techniques to minimize the network overhead and achieve effectiveness within the system. A special
class of wireless network, \ac{WANET},  which allowed nodes to communicate with each other without the need of
special infrastructure such as bridges and routers was developed. \ac{WANET} lead to use of wireless communication
for special applications with needs of distributed control. Shortly use of \ac{MANET} increased which allowed
continuos movement of the nodes. This was followed by use of wireless networking among vehicles to create \ac{VANET}
which allows communication of various parameters among vehicular focussed towards application for safety and cooperative
driving. The use of wireless networks in various domains has lead to a lot of research focussed towards improvements
and optimization which are often valid for all domains.

Clustering in wireless networks involves grouping nodes together which are geographically close to each other based on a
certain set of parameters. Parameter selection for clustering depends mostly on the type of application which would use
the clustered network. In \ac{VANET}s clustering of vehicles into groups provides a basis for limiting the networking
overhead and interference by efficiently defining the target nodes and designing filters to limit the traffic. Due to
the possibility of selecting huge range of parameters, there are a lot of solutions presented which target various
scenarios in the \ac{VANET}s. \cite{6256251} and \cite{BALI2014134} presents a detailed overview of research work in this
field in past years. In the following sections we would look at some of the important terminologies and parameters to create
a general overview of clustering in \ac{VANET} and help us to discuss and understand the methodologies better.

\section{Terminologies}

<Basic terminologies used in VANETs clustering>

\section{Clustering Parameters}


\chapter{Clustering methodologies}
\label{sec:methodologies}

<overview of clustering methodologies and criterions for clustering>

\section{Typical clustering steps}

\section{Method 1: Clustering using vehicular mobility}
\section{Method 2: Clustering using abstracted trajectory}
\section{Method 3: Clustering using vehicular density}
\section{Method 4: Hybrid clustering}


\chapter{Evaluation techniques}
\label{sec:evaluation}

<General simulation steps for evaluation>

\section{Important metrics for evaluation}



\chapter{Comparison of results}
\label{sec:comparison}



\chapter{Conclusion}
\label{sec:conclusions}

foo~\cite{6256251}
foo~\cite{ARKIAN2014197}
foo~\cite{6737622}
foo~\cite{6685518}
foo~\cite{4976256}
foo~\cite{5416361}
foo~\cite{5735785}
bar~\cite{6379136}
bar~\cite{HASSANABADI2014535}
bar~\cite{sommer2011bidirectionally}
bar~\cite{6407446}

\begin{itemize}
\item summarize again what your paper did, but now emphasize more the results, and comparisons
\item write conclusions that can be drawn from the results found and the discussion presented in the paper
\item future work (be very brief, explain what, but not much how, do not speculate about results or impact)
\item recommended length: one page.
\end{itemize}



\cleardoublepage

\listofabbreviations
\clearpage

\listoffigures
\clearpage

\listoftables
\clearpage

\printbibliography

\end{document}
