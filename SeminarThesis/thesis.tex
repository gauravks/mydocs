\documentclass[]{ccs-thesis}
% options:
% [germanthesis] - Thesis is written in German
% [plainunnumbered] - Don't print numbers on plain pages
% [earlydraft] - Settings for quick draft printouts
% [watermark] - Print current time/date at bottom of each page
% [phdthesis] - switch to PhD thesis style
% [twoside] - double sided
% [cutmargins] - text body fills complete page


% Author name. Separate multiple authors with commas.
\author{Gaurav Kumar Singh}
\birthday{10. November 1989}
\birthplace{Danapur}

% Title of your thesis.
\title{Clustering in Vehicular Networks}

% Choose one of the following lines. Feel free to change the word "Informatik" to match your degree program.
%\thesistype{Masterarbeit im Fach Informatik}\thesiscite{Master's Thesis~(Masterarbeit)}
%\thesistype{Bachelorarbeit im Fach Informatik}\thesiscite{Bachelor Thesis~(Bachelorarbeit)}
\thesistype{Seminararbeit im Fach Informatik}\thesiscite{Seminar Thesis~(Seminararbeit)}

% List of advisors, separated by commas.
\advisors{Christoph Sommer, Falko Dressler}

% List of referees, separated by commas.
\referees{Christoph Sommer, Falko Dressler}


% Define abbreviations used in the thesis here.
\acrodef{WSN}{Wireless Sensor Network}
\acrodef{MANET}{Mobile Ad Hoc Network}

\begin{document}

\pagenumbering{roman}

\maketitle

\chapter*{Abstract}
\addcontentsline{toc}{chapter}{Abstract}
\begin{otherlanguage*}{american}

This thesis captures an overview of ideas, techniques, results and future possibilities of clustering in vehicular networks.
Clustering is a technique to group nodes based on a selected criteria which defines certain level of similarities among the nodes.
Grouping the nodes together in such a way helps define or design a set of functionalities applicable only to the
group and can be applied to smaller sub-set. In a vehicular networking environment clustering presents possibilities
to group vehicles based on a parametr of interest and help to reduce the network traffic, achieve better network throughput,
effective information dissemination.

The thesis presents a set of parameter and respective algorithm based on them for vehicular networks as a comparitive study.
First chapter presents the motivation behind the clustering and outlines the basic set of problems which is presented by vehicular
Networks which the algorithm try to address. The second chapter describes the algorithms grouping them based on the main parameters
used for clustering. The third chapter captures a comparative study, highlighting benefits and drawbacks of different approaches based
on simulation results. Finally the thesis captures some ideas which will give an overview of the future research work on the topic.
 
 
\end{otherlanguage*}


\acresetall

\cleardoublepage
\tableofcontents

\cleardoublepage
\pagenumbering{arabic}

\chapter{Introduction}
%\chapter{Einleitung}
\label{sec:introduction}

\begin{itemize}
\item general motivation for your work, context and goals.
\item context: make sure to link where your work fits in
\item problem: gap in knowledge, too expensive, too slow, a deficiency, superseded technology
\item strategy: the way you will address the problem
\item recommended length: 1-2 pages.
\end{itemize}


\chapter{Fundamentals}
\label{sec:fundamentals}


\begin{itemize}
\item describe methods and techniques that build the basis of your work
\item include what's needed to understand your work (e.g., techniques, protocols, models, hardware, software, ...)
\item exclude what's not (e.g., anything you yourself did, anything your reader can be expected to know, ...)
\item review related work(!)
\item recommended length: approximately one third of the thesis.
\end{itemize}


\chapter{Developed architecture / System design / Implementation / ...}


\begin{itemize}
\item describe everything you yourself did (as opposed to the fundamentals chapter, which explains what you built on)
\item start with a theoretical approach
\item describe the developed system/algorithm/method from a high-level point of view
\item go ahead in presenting your developments in more detail
\item recommended length: approximately one third of the thesis.
\end{itemize}



\chapter{Evaluation}


\begin{itemize}
\item measurement setup / results / evaluation / discussion
\item whatever you have done, you must comment it, compare it to other systems, evaluate it
\item usually, adequate graphs help to show the benefits of your approach
\item each result/graph must not only be described, but also discussed (What's the reason for this peak? Why have you observed this effect? What does this tell about your architecture/system/implementation?)
\item recommended length: approximately one third of the thesis.
\end{itemize}



\chapter{Conclusion}


\begin{itemize}
\item summarize again what your paper did, but now emphasize more the results, and comparisons
\item write conclusions that can be drawn from the results found and the discussion presented in the paper
\item future work (be very brief, explain what, but not much how, do not speculate about results or impact)
\item recommended length: one page.
\end{itemize}



\cleardoublepage

\listofabbreviations
\clearpage

\listoffigures
\clearpage

\listoftables
\clearpage

\printbibliography

\end{document}
