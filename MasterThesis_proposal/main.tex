%% A simple template for a lab or course report using the Hagenberg setup
%% based on the standard LaTeX 'report' class
%% äöüÄÖÜß  <-- no German Umlauts here? Use an UTF-8 compatible editor!

%%% Magic comments for setting the correct parameters in compatible IDEs
% !TeX encoding = utf8
% !TeX program = pdflatex 
% !TeX spellcheck = en_US
% !BIB program = biber

\documentclass[english,notitlepage]{hgbreport}

\RequirePackage[utf8]{inputenc}		% remove when using lualatex oder xelatex!
\renewcommand{\chapter}[1]{}	% \chapter is deactivated
\renewcommand{\thesection}{\arabic{section}}
\setcounter{secnumdepth}{4}

\graphicspath{{images/}}   % where are the images?

\bibliography{references}   % requires file 'references.bib'
\ExecuteBibliographyOptions{backref=false}



%%-----------------------------------------------------------
\setcounter{chapter}{1}	% <----- set to assignment number!
%%-----------------------------------------------------------

\author{Gaurav Kumar Singh \\ \textit{\href{mailto:gauravks@mail.uni-paderborn.de}{gauravks@mail.uni-paderborn.de}}}
\title{Proposal for Master Thesis on topic \\
		Acceleration of Discontinuous Galerkin method \\
		using FPGAs communicating via IO channels}
\date{\today}


%%%----------------------------------------------------------
\begin{document}
%%%----------------------------------------------------------
\maketitle
%%%----------------------------------------------------------

\begin{abstract}\noindent
	Abstract giving a overview of the aim of the thesis
\end{abstract}

\section{Introduction}

FPGA based accelerators provides the flexibility to design application specific hardware accelerators and re-use the same
hardware for different kinds of problems. Such advantages has shown a increased popularity of the FPGA accelerators among
the high performance computing (HPC) researchers in recent times. FPGAs are mostly used to develop highly
optimized accelerators for common operations such as matrix multiplication, fast fourier transformation,
and sparse-matrix optimization which serve as the main computing blocks for many of the applications. These accelerators
mostly rely on the features such as pipelining for data re-use, replication for parallelization and memory localization
for lower memory latency.

The new Noctua high performance computing (HPC) cluster at Paderborn Center for Parallel Computing  (PC\textsuperscript{2})
which is equipped with additional Stratrix 10 FPGA accelerators provides an excellent opportunity to demonstrate the benefits
of using FPGA accelerators for applications from different domains. Along with the usual benefits mentioned previously, the
Nallatech 520N accelerator boards which are installed in the Noctua cluster, provides opportunities to use multiple
FPGA in parallel for the same application by using the four 100/40/25/10G QSFP28 network ports. This can be used to offload
the network communication required in some applications for data exchange while using MPI or similar distributed computing
algorithms such as the MIDG2\footnote{\url{https://github.com/tcew/MIDG2}}.

MIDG2 is a MPI based parallel computation implementation of the Discontinuous Galerikin (DG) \cite{hesthaven_nodal_2008} method.
DG method is a commonly used operation in many applications simulations to solve partial differential equations (PDE).
\textcite{kenter_opencl-based_2018} presents an openCL based FPGA accelerator which works for a single node and is 2x faster
when compared to a multithreaded CPU implementation. \textcite{kobayashi_opencl-ready_2018} has shown the benefits of using
FPGA network communication over the InfiBand network. Such implementation highlights the benefits of a FPGA based distributed
computing system which would be the mail goal of this thesis.

In the rest of the document, the objectives of the thesis would be defined introducing some of the components involved followed
by the timeline and initial outline of the thesis work.


\section{Objectives}

The Noctua being first of the academic HPC cluster equipped with FPGAs, we don't know of any implementation
utilizing FPGA in network configuration for more than two FPGAs. The proposed master thesis aims at presenting
such a system which evaluates the achievable acceleration with multiple FPGAs in point-to-point communication
in different network topologies using the MIDG2 as the base application for the evaluation. In respect to achieve
the goal following objectives is required to be achieved:

\begin{enumerate}
	\item \textbf{Objective 1}: Extend the openCL kernels of the MIDG2 FPGA implementation to use IO channels
	\item \textbf{Objective 2}: Identify and evaluate possible topologies using the 4 IO channels on Nallatech 520N boards
	\item \textbf{Objective 3}: Devise an effective partitioning scheme for the mesh to reduce communication overheads for the target topologies
\end{enumerate}

\subsection{Objective 1}

The first objective requires the understanding of following components

\subsubsection{Nodal Discontinuous Galerikin Method}	% title of a subtask

The nodal discontinuous Galerikin time domain method (DGTD) \cite{hesthaven_nodal_2008} is used to find solutions
for partial differential equations (PDE) numerically. The method is proved to be efficient in
producing results with computers as it relies on mathematical calculations on elemental basis.
This allows to perform computation in parallel on similar or different hardware helping to
solve problems from different domains quickly. DGTD method have particularly popular for applications
in the domains such as fluid mechanics, plasma physics and electrodynamics.

\textcite{Hesthaven_190449} presents the use of DGTD to solve time-domain Maxwell's equations with an 1D example.
These equations serve as the base for solving many of the electrodynamics problems and is adopted by
many computer based algorithms as well. \cite{} presents an algorithm implemented in C++ for computing
the fields using such equations in a simulation for K non-overlapping tetrahedra elements. As highlighted 
in \cite{} many other implementation have also been developed such as MIDG2
which has been ported for various platforms to utilize different kinds of hardware and software environments.
A extended version of MIDG2 which used openCL and MPI to provide acceleration will be used as the reference design.

\subsubsection{MIDG2 MPI FPGA implementation}
MIDG2 MPI FPGA is extended version of MIDG2 application utilizing openCL kernels to perform the
computation and MPI for communication between multiple nodes. The openCL kernels are
highly optimized to achieve maximum performance on a single FPGA using pipelining using
standard Intel channels. This implementation will be used as the reference design and
the existing architecture of the application is shown in figure .
The DG time domain computation are performed in time steps using multiple kernels.
Volume and the surface kernel is used to compute the volume and surface RHS field
contributions of the elements respectively. The RK kernel accumulates the RHS
field contribution computed in volume and surface kernel. Partial kernel reads in
the intermediate fields value of last time step from the FPGA memory and fills them
in a contiguous memory which is then distributed to the respective neighbors using
MPI asynchronous communication.

In order to achieve communication between multiple FPGAs, IO channels would be used
in the partial kernel to send the data directly to the neighboring FPGA.

\subsection{Objective 2}

The current Nallatech BSP for 520N supports 4 openCL IO channels based communication between
multiple FPGA using the 100/40/25/10G QSFP28 network ports. The second objective of
the thesis is to utilize these network ports to evaluate and identify a point-to-point
topology for the FPGA which is most effective for the MIDG2 implementation.

\subsubsection{IO Channels}

The openCL IO channels provided by the BSP are and extension of the standard Intel channels \cite{noauthor_intel_2018}
and can be used in similar way to provide blocking communication between multiple FPGAs.
Currents version of the channels support a maximum datarate of 40 Gbits/s using 256 bits per cycle.
As there is no MAC mechanism available yet, only point-to-point communication is possible between
two FPGAs.

\subsubsection{Proposed Topologies}

To identify a suitable topology for the application, multiple topologies
as shown in fig would be evaluated. The main criterions for the evaluation
would be latency and the overall bandwidth of the channels achieved in each
configuration. As the topology shall be know prior to execution of the application,
the topological information should be provided by a suitable topological file
to enable the FPGAs identify their neighbors.


\subsection{Objective 3}

The third objective is to identify a suitable partitioning configuration for the mesh elements so that
the distribution of the elements among the FPGAs utilizes the topological information. This is critical
as a good distribution scheme would reduce the overall network communication by placing neighboring elements
on neighboring nodes. The partitioning is currently done by the ParMETIS\footnote{\url{http://glaros.dtc.umn.edu/gkhome/metis/parmetis/overview}}
library. 

\subsubsection{ParMETIS}

ParMETIS is a MPI-based parallel library which is capable of partitioning a variety of unstructured graphs and meshes. MIDG2
uses the ParMETIS to partition the mesh elements among the MPI nodes. The partitioning api uses a set of parameters such as element
size, partitioning surface criteria and weights for distribution considering capabilities of the node. These parameters allows
ParMETIS to partition the mesh into section which reduces the overall shared elements among nodes and reduce the communication requirement.
The partitioning scheme would require to be updated in order to provide the ParMETIS library the information of the topology of the FPGA.

\section{Preliminary Outline}

\section{Proposed Time Schedule}


%%%----------------------------------------------------------

\section*{References}

\printbibliography[heading=noheader]

%%%----------------------------------------------------------

\end{document}
