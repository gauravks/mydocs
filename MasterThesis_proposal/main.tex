%% A simple template for a lab or course report using the Hagenberg setup
%% based on the standard LaTeX 'report' class
%% äöüÄÖÜß  <-- no German Umlauts here? Use an UTF-8 compatible editor!

%%% Magic comments for setting the correct parameters in compatible IDEs
% !TeX encoding = utf8
% !TeX program = pdflatex 
% !TeX spellcheck = en_US
% !BIB program = biber

\documentclass[english,notitlepage]{hgbreport}

\RequirePackage[utf8]{inputenc}		% remove when using lualatex oder xelatex!
\renewcommand{\chapter}[1]{}	% \chapter is deactivated
\renewcommand{\thesection}{\arabic{section}}
\setcounter{secnumdepth}{4}

\graphicspath{{images/}}   % where are the images?

\bibliography{references}   % requires file 'references.bib'
\ExecuteBibliographyOptions{backref=false}



%%-----------------------------------------------------------
\setcounter{chapter}{1}	% <----- set to assignment number!
%%-----------------------------------------------------------

\author{Gaurav Kumar Singh \\ \textit{\href{mailto:gauravks@mail.uni-paderborn.de}{gauravks@mail.uni-paderborn.de}}}
\title{Proposal for Master Thesis on topic \\
		Acceleration of Discontinuous Galerkin method using multiple
		array of FPGA accelerators}
\date{\today}


%%%----------------------------------------------------------
\begin{document}
%%%----------------------------------------------------------
\maketitle
%%%----------------------------------------------------------

\begin{abstract}\noindent
Abstract giving a overview of the aim of the thesis
\end{abstract}

\section{Introduction}

Motivation behind the thesis

\subsection{Discontinuous Galerkin Method}	% title of a subtask
Introduce the DG method to solve maxwell equation along with current implementation

\subsection{FPGA accelerators}
 
\section{Objectives}

\section{Preliminary Outline}

\section{Proposed Time Schedule}


%%%----------------------------------------------------------
  
\section*{References}

\printbibliography[heading=noheader]

%%%----------------------------------------------------------

\end{document}
