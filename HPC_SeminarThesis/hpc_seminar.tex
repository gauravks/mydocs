% Seminar paper for High-Performance Computing with FPGAs (SS2018)
% Gaurav Kumar Singh <gauravks@mail.uni-paderborn.de>
%
% based on a template by Holger Karl, (c) 2001

\documentclass[12pt,twoside]{article}
\usepackage{url}


%%%%%%%%%%%%%%%%%%%%%%%%%%%%%%%%%%%%%%%%%%%%%%%%%%%%%%%%%%%%%%%%%%%%%%%%%%%%%%
%%%%%%%%% configure these settings and you are good to go %%%%%%%%%%%%%%%%%%%%
%%%%%%%%%%%%%%%%%%%%%%%%%%%%%%%%%%%%%%%%%%%%%%%%%%%%%%%%%%%%%%%%%%%%%%%%%%%%%%
\newcommand{\participant}{Gaurav Kumar Singh}
\newcommand{\affiliation}{Paderborn University}
\urldef{\emailaddress}\url{gauravks@mail.uni-paderborn.de}
\newcommand{\topic}{Designs for accelerating Bioinformatic problem solving using FPGAs based HPC system}


%%%%%%%%%%%%%%%%%%%%%%%%%%%%%%%%%%%%%%%%%%%%%%%%%%%%%%%%%%%%%%%%%%%%%%%%%%%%%%
%%%%%%%%% don't make any other changes in the following   %%%%%%%%%%%%%%%%%%%%
%%%%%%%%% preamble unless you know what you are doing     %%%%%%%%%%%%%%%%%%%%
%%%%%%%%%%%%%%%%%%%%%%%%%%%%%%%%%%%%%%%%%%%%%%%%%%%%%%%%%%%%%%%%%%%%%%%%%%%%%%

\usepackage[english]{babel}
\usepackage[T1]{fontenc}
\usepackage[utf8]{inputenc}
\usepackage{times}
\RequirePackage[style=ieee,doi=true,isbn=false,url=false,mincitenames=1,maxcitenames=3,minbibnames=10,maxbibnames=10]{biblatex}
\usepackage{geometry}
\geometry{a4paper,body={15cm,22cm}}
\RequirePackage{xpatch}
\xpatchbibmacro{textcite}{\addspace}{\addnbspace}{}{}
\xpatchbibmacro{Textcite}{\addspace}{\addnbspace}{}{}

\usepackage{graphicx}
\usepackage{paralist}
\usepackage{fancyhdr}
\bibliography{references}
\pagestyle{fancy}
\fancyhead{}
\fancyhead[LE]{ \slshape \participant}
\fancyhead[LO]{}
\fancyhead[RE]{}
\fancyhead[RO]{ \slshape \topic}
\fancyfoot[C]{}

\begin{document}

\title{\topic}
\author{\Large{\participant}\\ \affiliation \\ {\small \emailaddress}}
\date{}
\maketitle
\thispagestyle{empty}

%%%%%%%%%%%%%%%%%%%%%%%%%%%%%%%%%%%%%%%%%%%%%%%%%%%%%%%%%%%%%%%%%%%%%%%%%%%%%%
%%%%%%%%% here starts the actual text                     %%%%%%%%%%%%%%%%%%%%
%%%%%%%%%%%%%%%%%%%%%%%%%%%%%%%%%%%%%%%%%%%%%%%%%%%%%%%%%%%%%%%%%%%%%%%%%%%%%%

% Abstract gives a brief summary of the main points of a paper:
\begin{abstract}

  The abstract to summarize

  \begin{enumerate}
	\item What Bioinformatic problem try to solve
	\item Why are they important to us (Diversity, uses)
	\item How FPGA based systems can help (Complexity, speed etc.)
\end{enumerate}
\end{abstract}

% the actual content, usually separated over a number of sections
% each section is assigned a label, in order to be able to put a
% crossreference to it

\section{Introduction}
\label{sec:introduction}

This section would introduce the topic of Bioinformaticss and HPC with FPGA and highlight the interdependence to create the base for the
rest of the paper.
\begin{enumerate}
	\item Some references to exiting known HPC systems used for solving Bioinformatic problems
	\item Introduce the structure of the thesis
\end{enumerate}

Reference for the section \cite{perez-sanchez_role_2014}, \cite{karanam_using_2006}

Also use a general example to capture general terminologies used in the domain

\section{Related Work}
\label{sec:relwork}

Should discuss if this should remain as a separate section or the related work can already be introduced in previous section?

\section{System Designs and Techniques}
\label{sec:designtech}

This section would give specific example for Techniques and related system designs which
use FPGA based HPC systems to accelerate the solution finding.

Add subsection to describe the techniques along with references.

\subsection{Smith-Waterman Algorithm}

References \cite{rucci_smith-waterman_2014}, \cite{smith_identification_1981}

\subsection{BLAST}
\label{sec:blast}

Reference \cite{mahram_fast_2010}, \cite{altschul_basic_1990}

\subsection{HMMER}

References \cite{oliver_integrating_2008} \cite{abbas_combining_2015}

\subsection{RIVYERA}

Reference \cite{wienbrandt_improvement_2012}, \cite{wienbrandt_fpga-based_2014}
\section{Evaluation and comparison}
\label{sec:eval}

This section should highlight the possible acceleration which is possible with the FPGA based system
using the Evaluation data of different techniques and present a comparative study of how this techniques
vary to each other and to traditional HPC and serial computing.

This should be able to highlight the advantages of using FPGAs for certain problem to reduce cost and time for
for the problems.

\section{Conclusion}
\label{sec:concl}

%% the following commands include the biliographic information (in BibTeX format) from the
%% file template.bib

\printbibliography

\end{document}
