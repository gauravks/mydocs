% Seminar paper for High-Performance Computing with FPGAs (SS2018)
% Gaurav Kumar Singh <gauravks@mail.uni-paderborn.de>
%
% based on a template by Holger Karl, (c) 2001

\documentclass[12pt,twoside]{article}
\usepackage{url}


%%%%%%%%%%%%%%%%%%%%%%%%%%%%%%%%%%%%%%%%%%%%%%%%%%%%%%%%%%%%%%%%%%%%%%%%%%%%%%
%%%%%%%%% configure these settings and you are good to go %%%%%%%%%%%%%%%%%%%%
%%%%%%%%%%%%%%%%%%%%%%%%%%%%%%%%%%%%%%%%%%%%%%%%%%%%%%%%%%%%%%%%%%%%%%%%%%%%%%
\newcommand{\participant}{Gaurav Kumar Singh}
\newcommand{\affiliation}{Paderborn University}
\urldef{\emailaddress}\url{gauravks@mail.uni-paderborn.de}
\newcommand{\topic}{Designs for accelerating Bioinformatic problem solving using FPGAs based HPC system}


%%%%%%%%%%%%%%%%%%%%%%%%%%%%%%%%%%%%%%%%%%%%%%%%%%%%%%%%%%%%%%%%%%%%%%%%%%%%%%
%%%%%%%%% don't make any other changes in the following   %%%%%%%%%%%%%%%%%%%%
%%%%%%%%% preamble unless you know what you are doing     %%%%%%%%%%%%%%%%%%%%
%%%%%%%%%%%%%%%%%%%%%%%%%%%%%%%%%%%%%%%%%%%%%%%%%%%%%%%%%%%%%%%%%%%%%%%%%%%%%%

\usepackage[english]{babel}
\usepackage[T1]{fontenc}
\usepackage[utf8]{inputenc}
\usepackage{times}
\RequirePackage[style=ieee,doi=true,isbn=false,url=false,mincitenames=1,maxcitenames=3,minbibnames=10,maxbibnames=10]{biblatex}
\DeclareDatamodelFields[type=list,datatype=isbn]{isbn}
\usepackage{geometry}
\geometry{a4paper,body={15cm,22cm}}
\RequirePackage{xpatch}
\xpatchbibmacro{textcite}{\addspace}{\addnbspace}{}{}
\xpatchbibmacro{Textcite}{\addspace}{\addnbspace}{}{}

\usepackage{graphicx}
\usepackage{paralist}
\usepackage{fancyhdr}
\bibliography{references}
\pagestyle{fancy}
\fancyhead{}
\fancyhead[LE]{ \slshape \participant}
\fancyhead[LO]{}
\fancyhead[RE]{}
\fancyhead[RO]{ \slshape \topic}
\fancyfoot[C]{}

\begin{document}

\title{\topic}
\author{\Large{\participant}\\ \affiliation \\ {\small \emailaddress}}
\date{}
\maketitle
\thispagestyle{empty}

%%%%%%%%%%%%%%%%%%%%%%%%%%%%%%%%%%%%%%%%%%%%%%%%%%%%%%%%%%%%%%%%%%%%%%%%%%%%%%
%%%%%%%%% here starts the actual text                     %%%%%%%%%%%%%%%%%%%%
%%%%%%%%%%%%%%%%%%%%%%%%%%%%%%%%%%%%%%%%%%%%%%%%%%%%%%%%%%%%%%%%%%%%%%%%%%%%%%

% Abstract gives a brief summary of the main points of a paper:
\begin{abstract}

 The Human Genome project was marked as completed in the year 2003 which opened vast avenues 
 for research towards developing and enhancing Genomic analysis techniques. With such vast sequence
 database available, the Genomic research greatly depends on bioinformatics capabilities and improvements
 to the computational speed would help in analysis the data to discover causes and treatments for various
 diseases. This has been a major driving factor to develop techniques to increase the  processing
 capabilities of the  existing algorithms, tools and techniques by utilizing advancements in computing power.
 FPGA based acceleration for existing algorithms presents very promising advantages, reducing processing times
 by huge factor compared to other CPU and GPU based techniques. Similarly, the introduction of high performance
 clusters to distribute the processing has already been used and shown to be effective for large sequence analysis.

 Combining these technologies together possess great benefits for even speeding up the analysis of the huge databases.
 This paper will present some of the techniques and heterogenous system which have been developed and utilized to
 speed the the genome analysis from years to days. Initially we look at bioinformatics application areas where
 FPGA and HPC system are beneficial. Then the paper describes some of the tools and algorthmic accelerators
 using FPGA and HPC in a heterogenous system. The last part presents some systems and evaluation
 results in terms of speedup compared to existing systems and tools.

\end{abstract}

% the actual content, usually separated over a number of sections
% each section is assigned a label, in order to be able to put a
% crossreference to it

\section{Introduction}
\label{sec:introduction}

Humans quest to understand the basic biological processes lead to development of research areas such as
biochemistry and biotechnology. Multiple decades of research in the biological molecules helped us in
understanding the existence DNA and genome which defines how a living organism behaves and exist.
On the other hand the advancement in computer technologies and increased use of them in healthcare, biomedical
and computational biological research has helped find cure and medical treatments for many complex health
issues and save many lives over the years.

In efforts to increase the knowledge of genomes, the field of Bioinformatic was created. Bioinformatic majorly involves
the study of biological molecules (biomolecules) which build up the cells of the living organisms. As with the other biological fields,
bioinformatic aims at utilizing the capabilities of the computer science to build and analyse molecular sequences (genes) of genomes.
In this direction, The \emph{Human Genome Project}\footnote{\url{http://https://www.genome.gov/}}  was started in late 1990 and was
completed in 2003 successfully. "A 2.91-billion base pair (bp) consensus sequence of the euchromatic portion of
the human genome was generated by the whole-genome shotgun sequencing method"\cite{venter_sequence_2001}. This was a huge
step but also presented the problem of huge processing times for analysis of such a large database of genome for extracting any
useful information. The existing algorithms for database searches such as Smith-Waterman \cite{smith_identification_1981} based
on dynamic programming for local similarity estimation and heuristics based BLAST \cite{altschul_basic_1990} were limited by
high computation times. \textcite{schmidt_massively_2002} have demonstrated a parallel system which helps to speed up the 
molecular sequence analyse. The main limiting factors at this point were the processing capabilities of the computing units
on which the algorithms were running.

{Add pic of human genome from somewhere}

Various methods in the past decades have evolved to provide higher computing and data processing capabilities for various application domain.
The earliest method being hardware acceleration provided by symmetric multiprocessing which allows distribution of computing to different processor
sharing a common memory. The next major acceleration achieved has been the development of high performance computing clusters (HPC).  HPC
system work by splitting and distributing the problem over multiple similar processing units popularly known as nodes. Each of the node,
consists of a high performance processor with multiple cores and sharing the same common memory. The nodes in the clusters are connected to
each other with high speed Ethernet connections for exchange of data and control information as shown in fig<>. Each node can be used to
process a sub-set of the data parallely decreasing the overall computing time for the problem. Due to such benefits, these techniques have
being used to speed up the Bioinformatic algorithms by modification to work on these HPC clusters and utilize the benefits.
\cite{boukerche_parallel_2005,martins_multithreaded_2000} gives implementation of the famous Smith-Waterman algorithm on HPC systems.
A various number of parallel implementation for BLAST such as mpiBLAST \cite{darling_design_2003} are available as well which prove
to be more time efficient.

{Add pic of cluster} 

Another step in increasing the processing capabilities of the clusters is use of GPU. GPUs allows offloading
the vector based arithmetic operations for large datasets. The GPUs prove to be excellent accelerators for reducing
the processing time of complex calculation on large amount of data which are common in many of the application domain
utilizing the clusters. \textcite{liu_cuda-blastp_2011} have presented such a system which is capable of performing
10 times faster compared to serial versions.

Though the parallel implementation with CPU and GPU help in achieving faster processing time, its heavily dependant
on the size of the cluster. Also the speedup is dependant on the size of the problem as well. These reasons
made researchers to look for areas for improving the execution times of the algorithm by using hardware based
accelerators for the algorithms to reduce processing time for each operation. This is where the FPGA has 
helped a lot by providing opportunities to implement the algorithms directly in the hardware. The flexibility
of FPGA based accelerators makes them very useful to design application specific acceleration hardware and
also re-use them for different kinds of problems. Currently a lot of accelerators are available which the bioinformatic
community is benefiting from. This paper would discuss some of these implementation and give an overview of how
such accelerators are integrated with the HPC clusters to build heterogenous systems which are used to
achieve very high processing speeds to reduce the time from days to hours required for some bioinformatic application.

The rest of the paper is divided into 3 sections. Section 2 introduces the bioinformatic application domain giving
details of algorithms and tools popularly used. Section 3 will present the optimization techniques for genome
comparison by FPGA and heterogenous systems and the last section presents results achieved by such optimization
for some of the current systems.

\section {Bioinformatic}

Introduction to the concept of Bioinformatic

\subsection{Application areas}
\subsubsection{Genome Assembly}
\subsubsection{Contenct-based Search}
\subsubsection{Genome comparison}
\subsubsection{Pattern Matching}
\subsubsection{Genome Databases}

\subsection{Algorithms}
\subsubsection{Dynamic Programming}
References \cite{rucci_smith-waterman_2014}, \cite{smith_identification_1981}
\subsubsection{Seed-based Heuristics}
Reference \cite{mahram_fast_2010}, \cite{altschul_basic_1990}

\subsubsection{Languages Models and Profiles}
References \cite{oliver_integrating_2008} \cite{abbas_combining_2015}
\section{Optimization of Genome comparison algorithms}
\label{sec:designtech}
This section would give specific example for Techniques and related system designs which
use FPGA based HPC systems to accelerate the solution finding.

Add subsection to describe the techniques along with references.

\subsection{FPGA accelerators for Smith-Waterman}
\subsection{Heterogenous system designs}

\section{Evaluation and comparison}
\label{sec:eval}

This section should highlight the possible acceleration which is possible with the FPGA based system
using the Evaluation data of different techniques and present a comparative study of how this techniques
vary to each other and to traditional HPC and serial computing.

This should be able to highlight the advantages of using FPGAs for certain problem to reduce cost and time for
for the problems.

\section{Conclusion}
\label{sec:concl}

Timelogic accelerators
%% the following commands include the biliographic information (in BibTeX format) from the
%% file template.bib

\printbibliography

\end{document}
